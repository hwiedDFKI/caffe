

 \subsection*{title\+: Multi-\/\+G\+PU Usage, Hardware Configuration Assumptions, and Performance }

\section*{Multi-\/\+G\+PU Usage}

Currently Multi-\/\+G\+PU is only supported via the C/\+C++ paths and only for training.

The G\+P\+Us to be used for training can be set with the \char`\"{}-\/gpu\char`\"{} flag on the command line to the \textquotesingle{}caffe\textquotesingle{} tool. e.\+g. \char`\"{}build/tools/caffe train -\/-\/solver=models/bvlc\+\_\+alexnet/solver.\+prototxt -\/-\/gpu=0,1\char`\"{} will train on G\+P\+Us 0 and 1.

{\bfseries N\+O\+TE}\+: each G\+PU runs the batchsize specified in your train\+\_\+val.\+prototxt. So if you go from 1 G\+PU to 2 G\+PU, your effective batchsize will double. e.\+g. if your train\+\_\+val.\+prototxt specified a batchsize of 256, if you run 2 G\+P\+Us your effective batch size is now 512. So you need to adjust the batchsize when running multiple G\+P\+Us and/or adjust your solver params, specifically learning rate.

\section*{Hardware Configuration Assumptions}

The current implementation uses a tree reduction strategy. e.\+g. if there are 4 G\+P\+Us in the system, 0\+:1, 2\+:3 will exchange gradients, then 0\+:2 (top of the tree) will exchange gradients, 0 will calculate updated model, 0-\/$>$2, and then 0-\/$>$1, 2-\/$>$3.

For best performance, P2P D\+MA access between devices is needed. Without P2P access, for example crossing P\+C\+Ie root complex, data is copied through host and effective exchange bandwidth is greatly reduced.

Current implementation has a \char`\"{}soft\char`\"{} assumption that the devices being used are homogeneous. In practice, any devices of the same general class should work together, but performance and total size is limited by the smallest device being used. e.\+g. if you combine a TitanX and a G\+T\+X980, performance will be limited by the 980. Mixing vastly different levels of boards, e.\+g. Kepler and Fermi, is not supported.

\char`\"{}nvidia-\/smi topo -\/m\char`\"{} will show you the connectivity matrix. You can do P2P through P\+C\+Ie bridges, but not across socket level links at this time, e.\+g. across C\+PU sockets on a multi-\/socket motherboard.

\section*{Scaling Performance}

Performance is {\bfseries heavily} dependent on the P\+C\+Ie topology of the system, the configuration of the neural network you are training, and the speed of each of the layers. Systems like the D\+I\+G\+I\+TS Dev\+Box have an optimized P\+C\+Ie topology (X99-\/E WS chipset). In general, scaling on 2 G\+P\+Us tends to be $\sim$1.8X on average for networks like Alex\+Net, Caffe\+Net, V\+GG, Google\+Net. 4 G\+P\+Us begins to have falloff in scaling. Generally with \char`\"{}weak scaling\char`\"{} where the batchsize increases with the number of G\+P\+Us you will see 3.\+5x scaling or so. With \char`\"{}strong scaling\char`\"{}, the system can become communication bound, especially with layer performance optimizations like those in \href{http://nvidia.com/cudnn}{\tt cu\+D\+N\+Nv3}, and you will likely see closer to mid 2.\+x scaling in performance. Networks that have heavy computation compared to the number of parameters tend to have the best scaling performance. 