

 title\+: Feature extraction with Caffe C++ code. description\+: Extract Caffe\+Net / Alex\+Net features using the Caffe utility. category\+: example include\+\_\+in\+\_\+docs\+: true \subsection*{priority\+: 10 }

\section*{Extracting Features }

In this tutorial, we will extract features using a pre-\/trained model with the included C++ utility. Note that we recommend using the Python interface for this task, as for example in the \href{http://nbviewer.ipython.org/github/BVLC/caffe/blob/master/examples/00-classification.ipynb}{\tt filter visualization example}.

Follow instructions for \href{../../installation.html}{\tt installing Caffe} and run {\ttfamily scripts/download\+\_\+model\+\_\+binary.\+py models/bvlc\+\_\+reference\+\_\+caffenet} from caffe root directory. If you need detailed information about the tools below, please consult their source code, in which additional documentation is usually provided.

\subsection*{Select data to run on }

We\textquotesingle{}ll make a temporary folder to store things into. \begin{DoxyVerb}mkdir examples/_temp
\end{DoxyVerb}


Generate a list of the files to process. We\textquotesingle{}re going to use the images that ship with caffe. \begin{DoxyVerb}find `pwd`/examples/images -type f -exec echo {} \; > examples/_temp/temp.txt
\end{DoxyVerb}


The {\ttfamily Image\+Data\+Layer} we\textquotesingle{}ll use expects labels after each filenames, so let\textquotesingle{}s add a 0 to the end of each line \begin{DoxyVerb}sed "s/$/ 0/" examples/_temp/temp.txt > examples/_temp/file_list.txt
\end{DoxyVerb}


\subsection*{Define the Feature Extraction Network Architecture }

In practice, subtracting the mean image from a dataset significantly improves classification accuracies. Download the mean image of the I\+L\+S\+V\+RC dataset. \begin{DoxyVerb}./data/ilsvrc12/get_ilsvrc_aux.sh
\end{DoxyVerb}


We will use {\ttfamily data/ilsvrc212/imagenet\+\_\+mean.\+binaryproto} in the network definition prototxt.

Let\textquotesingle{}s copy and modify the network definition. We\textquotesingle{}ll be using the {\ttfamily Image\+Data\+Layer}, which will load and resize images for us. \begin{DoxyVerb}cp examples/feature_extraction/imagenet_val.prototxt examples/_temp
\end{DoxyVerb}


\subsection*{Extract Features }

Now everything necessary is in place. \begin{DoxyVerb}./build/tools/extract_features.bin models/bvlc_reference_caffenet/bvlc_reference_caffenet.caffemodel examples/_temp/imagenet_val.prototxt fc7 examples/_temp/features 10 leveldb
\end{DoxyVerb}


The name of feature blob that you extract is {\ttfamily fc7}, which represents the highest level feature of the reference model. We can use any other layer, as well, such as {\ttfamily conv5} or {\ttfamily pool3}.

The last parameter above is the number of data mini-\/batches.

The features are stored to Level\+DB {\ttfamily examples/\+\_\+temp/features}, ready for access by some other code.

If you meet with the error \char`\"{}\+Check failed\+: status.\+ok() Failed to open leveldb examples/\+\_\+temp/features\char`\"{}, it is because the directory examples/\+\_\+temp/features has been created the last time you run the command. Remove it and run again. \begin{DoxyVerb}rm -rf examples/_temp/features/
\end{DoxyVerb}


If you\textquotesingle{}d like to use the Python wrapper for extracting features, check out the \href{http://nbviewer.ipython.org/github/BVLC/caffe/blob/master/examples/00-classification.ipynb}{\tt filter visualization notebook}.

\subsection*{Clean Up }

Let\textquotesingle{}s remove the temporary directory now. \begin{DoxyVerb}rm -r examples/_temp\end{DoxyVerb}
 