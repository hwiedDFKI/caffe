

 title\+: Caffe\+Net C++ Classification example description\+: A simple example performing image classification using the low-\/level C++ A\+PI. category\+: example include\+\_\+in\+\_\+docs\+: true \subsection*{priority\+: 10 }

\section*{Classifying Image\+Net\+: using the C++ A\+PI}

Caffe, at its core, is written in C++. It is possible to use the C++ A\+PI of Caffe to implement an image classification application similar to the Python code presented in one of the Notebook examples. To look at a more general-\/purpose example of the Caffe C++ A\+PI, you should study the source code of the command line tool {\ttfamily caffe} in {\ttfamily tools/caffe.\+cpp}.

\subsection*{Presentation}

A simple C++ code is proposed in {\ttfamily examples/cpp\+\_\+classification/classification.\+cpp}. For the sake of simplicity, this example does not support oversampling of a single sample nor batching of multiple independent samples. This example is not trying to reach the maximum possible classification throughput on a system, but special care was given to avoid unnecessary pessimization while keeping the code readable.

\subsection*{Compiling}

The C++ example is built automatically when compiling Caffe. To compile Caffe you should follow the documented instructions. The classification example will be built as {\ttfamily examples/classification.\+bin} in your build directory.

\subsection*{Usage}

To use the pre-\/trained Caffe\+Net model with the classification example, you need to download it from the \char`\"{}\+Model Zoo\char`\"{} using the following script\+: 
\begin{DoxyCode}
./scripts/download\_model\_binary.py models/bvlc\_reference\_caffenet
\end{DoxyCode}
 The Image\+Net labels file (also called the {\itshape synset file}) is also required in order to map a prediction to the name of the class\+: 
\begin{DoxyCode}
./data/ilsvrc12/get\_ilsvrc\_aux.sh
\end{DoxyCode}
 Using the files that were downloaded, we can classify the provided cat image ({\ttfamily examples/images/cat.\+jpg}) using this command\+: 
\begin{DoxyCode}
./build/examples/cpp\_classification/classification.bin \(\backslash\)
  models/bvlc\_reference\_caffenet/deploy.prototxt \(\backslash\)
  models/bvlc\_reference\_caffenet/bvlc\_reference\_caffenet.caffemodel \(\backslash\)
  data/ilsvrc12/imagenet\_mean.binaryproto \(\backslash\)
  data/ilsvrc12/synset\_words.txt \(\backslash\)
  examples/images/cat.jpg
\end{DoxyCode}
 The output should look like this\+: 
\begin{DoxyCode}
---------- Prediction for examples/images/cat.jpg ----------
0.3134 - "n02123045 tabby, tabby cat"
0.2380 - "n02123159 tiger cat"
0.1235 - "n02124075 Egyptian cat"
0.1003 - "n02119022 red fox, Vulpes vulpes"
0.0715 - "n02127052 lynx, catamount"
\end{DoxyCode}


\subsection*{Improving Performance}

To further improve performance, you will need to leverage the G\+PU more, here are some guidelines\+:


\begin{DoxyItemize}
\item Move the data on the G\+PU early and perform all preprocessing operations there.
\item If you have many images to classify simultaneously, you should use batching (independent images are classified in a single forward pass).
\item Use multiple classification threads to ensure the G\+PU is always fully utilized and not waiting for an I/O blocked C\+PU thread. 
\end{DoxyItemize}