

 \subsection*{title\+: Slice Layer }

\section*{Slice Layer}


\begin{DoxyItemize}
\item Layer type\+: {\ttfamily Slice}
\item \href{http://caffe.berkeleyvision.org/doxygen/classcaffe_1_1SliceLayer.html}{\tt Doxygen Documentation}
\item Header\+: \href{https://github.com/BVLC/caffe/blob/master/include/caffe/layers/slice_layer.hpp}{\tt {\ttfamily ./include/caffe/layers/slice\+\_\+layer.hpp}}
\item C\+PU implementation\+: \href{https://github.com/BVLC/caffe/blob/master/src/caffe/layers/slice_layer.cpp}{\tt {\ttfamily ./src/caffe/layers/slice\+\_\+layer.cpp}}
\item C\+U\+DA G\+PU implementation\+: \href{https://github.com/BVLC/caffe/blob/master/src/caffe/layers/slice_layer.cu}{\tt {\ttfamily ./src/caffe/layers/slice\+\_\+layer.cu}}
\end{DoxyItemize}

The {\ttfamily Slice} layer is a utility layer that slices an input layer to multiple output layers along a given dimension (currently num or channel only) with given slice indices.


\begin{DoxyItemize}
\item Sample \begin{DoxyVerb}layer {
  name: "slicer_label"
  type: "Slice"
  bottom: "label"
  ## Example of label with a shape N x 3 x 1 x 1
  top: "label1"
  top: "label2"
  top: "label3"
  slice_param {
    axis: 1
    slice_point: 1
    slice_point: 2
  }
}
\end{DoxyVerb}

\end{DoxyItemize}

{\ttfamily axis} indicates the target axis; {\ttfamily slice\+\_\+point} indicates indexes in the selected dimension (the number of indices must be equal to the number of top blobs minus one).

\subsection*{Parameters}


\begin{DoxyItemize}
\item Parameters ({\ttfamily Slice\+Parameter slice\+\_\+param})
\item From \href{https://github.com/BVLC/caffe/blob/master/src/caffe/proto/caffe.proto}{\tt {\ttfamily ./src/caffe/proto/caffe.proto}}\+:
\end{DoxyItemize}

\{\% highlight Protobuf \%\} \{\% include proto/\+Slice\+Parameter.\+txt \%\} \{\% endhighlight \%\} 