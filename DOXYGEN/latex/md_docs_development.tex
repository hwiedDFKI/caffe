

 \subsection*{title\+: Developing and Contributing }

\section*{Development and Contributing}

Caffe is developed with active participation of the community.~\newline
 The \href{http://bair.berkeley.edu/}{\tt B\+A\+IR}/\+B\+V\+LC brewers welcome all contributions!

The exact details of contributions are recorded by versioning and cited in our \href{http://caffe.berkeleyvision.org/#acknowledgements}{\tt acknowledgements}. This method is impartial and always up-\/to-\/date.

\subsection*{License}

Caffe is licensed under the terms in \href{https://github.com/BVLC/caffe/blob/master/LICENSE}{\tt L\+I\+C\+E\+N\+SE}. By contributing to the project, you agree to the license and copyright terms therein and release your contribution under these terms.

\subsection*{Copyright}

Caffe uses a shared copyright model\+: each contributor holds copyright over their contributions to Caffe. The project versioning records all such contribution and copyright details.

If a contributor wants to further mark their specific copyright on a particular contribution, they should indicate their copyright solely in the commit message of the change when it is committed. Do not include copyright notices in files for this purpose.

\subsubsection*{Documentation}

This website, written with \href{http://jekyllrb.com/}{\tt Jekyll}, acts as the official Caffe documentation -- simply run {\ttfamily scripts/build\+\_\+docs.\+sh} and view the website at {\ttfamily \href{http://0.0.0.0:4000}{\tt http\+://0.\+0.\+0.\+0\+:4000}}.

We prefer tutorials and examples to be documented close to where they live, in {\ttfamily readme.\+md} files. The {\ttfamily build\+\_\+docs.\+sh} script gathers all {\ttfamily examples/$\ast$$\ast$/readme.md} and {\ttfamily examples/$\ast$.ipynb} files, and makes a table of contents. To be included in the docs, the readme files must be annotated with \href{http://jekyllrb.com/docs/frontmatter/}{\tt Y\+A\+ML front-\/matter}, including the flag {\ttfamily include\+\_\+in\+\_\+docs\+: true}. Similarly for I\+Python notebooks\+: simply include {\ttfamily \char`\"{}include\+\_\+in\+\_\+docs\char`\"{}\+: true} in the {\ttfamily \char`\"{}metadata\char`\"{}} J\+S\+ON field.

Other docs, such as installation guides, are written in the {\ttfamily docs} directory and manually linked to from the {\ttfamily index.\+md} page.

We strive to provide lots of usage examples, and to document all code in docstrings. We absolutely appreciate any contribution to this effort!

\subsubsection*{Versioning}

The {\ttfamily master} branch receives all new development including community contributions. We try to keep it in a reliable state, but it is the bleeding edge, and things do get broken every now and then. B\+A\+IR maintainers will periodically make releases by marking stable checkpoints as tags and maintenance branches. \href{https://github.com/BVLC/caffe/releases}{\tt Past releases} are catalogued online.

\paragraph*{Issues \& Pull Request Protocol}

Post \href{https://github.com/BVLC/caffe/issues}{\tt Issues} to propose features, report \href{https://github.com/BVLC/caffe/issues?labels=bug&page=1&state=open}{\tt bugs}, and discuss framework code. Large-\/scale development work is guided by \href{https://github.com/BVLC/caffe/issues?milestone=1}{\tt milestones}, which are sets of Issues selected for bundling as releases.

Please note that since the core developers are largely researchers, we may work on a feature in isolation for some time before releasing it to the community, so as to claim honest academic contribution. We do release things as soon as a reasonable technical report may be written, and we still aim to inform the community of ongoing development through Github Issues.

{\bfseries When you are ready to develop a feature or fixing a bug, follow this protocol}\+:


\begin{DoxyItemize}
\item Develop in \href{https://www.atlassian.com/git/workflows#!workflow-feature-branch}{\tt feature branches} with descriptive names. Branch off of the latest {\ttfamily master}.
\item Bring your work up-\/to-\/date by \href{http://git-scm.com/book/en/Git-Branching-Rebasing}{\tt rebasing} onto the latest {\ttfamily master} when done. (Groom your changes by \href{https://help.github.com/articles/interactive-rebase}{\tt interactive rebase}, if you\textquotesingle{}d like.)
\item \href{https://help.github.com/articles/using-pull-requests}{\tt Pull request} your contribution to {\ttfamily B\+V\+L\+C/caffe}\textquotesingle{}s {\ttfamily master} branch for discussion and review.
\begin{DoxyItemize}
\item Make P\+Rs {\itshape as soon as development begins}, to let discussion guide development.
\item A PR is only ready for merge review when it is a fast-\/forward merge, and all code is documented, linted, and tested -- that means your PR must include tests!
\end{DoxyItemize}
\item When the PR satisfies the above properties, use comments to request maintainer review.
\end{DoxyItemize}

The following is a poetic presentation of the protocol in code form.

\paragraph*{\href{https://github.com/shelhamer}{\tt Shelhamer\textquotesingle{}s} “life of a branch in four acts”}

Make the {\ttfamily feature} branch off of the latest {\ttfamily bvlc/master} \begin{DoxyVerb}git checkout master
git pull upstream master
git checkout -b feature
# do your work, make commits
\end{DoxyVerb}


Prepare to merge by rebasing your branch on the latest {\ttfamily bvlc/master} \begin{DoxyVerb}# make sure master is fresh
git checkout master
git pull upstream master
# rebase your branch on the tip of master
git checkout feature
git rebase master
\end{DoxyVerb}


Push your branch to pull request it into {\ttfamily B\+V\+L\+C/caffe\+:master} \begin{DoxyVerb}git push origin feature
# ...make pull request to master...
\end{DoxyVerb}


Now make a pull request! You can do this from the command line ({\ttfamily git pull-\/request -\/b master}) if you install \href{https://github.com/github/hub}{\tt hub}. Hub has many other magical uses.

The pull request of {\ttfamily feature} into {\ttfamily master} will be a clean merge. Applause.

{\bfseries Historical note}\+: Caffe once relied on a two branch {\ttfamily master} and {\ttfamily dev} workflow. P\+Rs from this time are still open but these will be merged into {\ttfamily master} or closed.

\subsubsection*{Testing}

Run {\ttfamily make runtest} to check the project tests. New code requires new tests. Pull requests that fail tests will not be accepted.

The {\ttfamily gtest} framework we use provides many additional options, which you can access by running the test binaries directly. One of the more useful options is {\ttfamily -\/-\/gtest\+\_\+filter}, which allows you to filter tests by name\+: \begin{DoxyVerb}# run all tests with CPU in the name
build/test/test_all.testbin --gtest_filter='*CPU*'

# run all tests without GPU in the name (note the leading minus sign)
build/test/test_all.testbin --gtest_filter=-'*GPU*'
\end{DoxyVerb}


To get a list of all options {\ttfamily googletest} provides, simply pass the {\ttfamily -\/-\/help} flag\+: \begin{DoxyVerb}build/test/test_all.testbin --help
\end{DoxyVerb}


\subsubsection*{Style}


\begin{DoxyItemize}
\item {\bfseries Run {\ttfamily make lint} to check C++ code.}
\item Wrap lines at 80 chars.
\item Follow \href{https://google.github.io/styleguide/cppguide.html}{\tt Google C++ style} and \href{https://google.github.io/styleguide/pyguide.html}{\tt Google python style} + \href{http://legacy.python.org/dev/peps/pep-0008/}{\tt P\+EP 8}.
\item Remember that “a foolish consistency is the hobgoblin of little minds,” so use your best judgement to write the clearest code for your particular case. 
\end{DoxyItemize}