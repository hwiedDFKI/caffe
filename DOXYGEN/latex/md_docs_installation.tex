

 \subsection*{title\+: Installation }

\section*{Installation}

Prior to installing, have a glance through this guide and take note of the details for your platform. We install and run Caffe on Ubuntu 16.\+04–12.04, OS X 10.\+11–10.8, and through Docker and A\+WS. The official Makefile and {\ttfamily Makefile.\+config} build are complemented by a \href{#cmake-build}{\tt community C\+Make build}.

{\bfseries Step-\/by-\/step Instructions}\+:


\begin{DoxyItemize}
\item \href{https://github.com/BVLC/caffe/tree/master/docker}{\tt Docker setup} {\itshape out-\/of-\/the-\/box brewing}
\item \href{install_apt.html}{\tt Ubuntu installation} {\itshape the standard platform}
\item \href{install_apt_debian.html}{\tt Debian installation} {\itshape install caffe with a single command}
\item \href{install_osx.html}{\tt OS X installation}
\item \href{install_yum.html}{\tt R\+H\+EL / Cent\+OS / Fedora installation}
\item \href{https://github.com/BVLC/caffe/tree/windows}{\tt Windows} {\itshape see the Windows branch led by Guillaume Dumont}
\item \href{https://github.com/BVLC/caffe/tree/opencl}{\tt Open\+CL} {\itshape see the Open\+CL branch led by Fabian Tschopp}
\item \href{https://aws.amazon.com/marketplace/pp/B01M0AXXQB}{\tt A\+WS A\+MI} {\itshape official deep learning amazon machine image from A\+WS}
\end{DoxyItemize}

{\bfseries Overview}\+:


\begin{DoxyItemize}
\item \href{#prerequisites}{\tt Prerequisites}
\item \href{#compilation}{\tt Compilation}
\item \href{#hardware}{\tt Hardware}
\end{DoxyItemize}

When updating Caffe, it\textquotesingle{}s best to {\ttfamily make clean} before re-\/compiling.

\subsection*{Prerequisites}

Caffe has several dependencies\+:


\begin{DoxyItemize}
\item \href{https://developer.nvidia.com/cuda-zone}{\tt C\+U\+DA} is required for G\+PU mode.
\begin{DoxyItemize}
\item library version 7+ and the latest driver version are recommended, but 6.$\ast$ is fine too
\item 5.\+5, and 5.\+0 are compatible but considered legacy
\end{DoxyItemize}
\item \href{http://en.wikipedia.org/wiki/Basic_Linear_Algebra_Subprograms}{\tt B\+L\+AS} via A\+T\+L\+AS, M\+KL, or Open\+B\+L\+AS.
\item \href{http://www.boost.org/}{\tt Boost} $>$= 1.\+55
\item {\ttfamily protobuf}, {\ttfamily glog}, {\ttfamily gflags}, {\ttfamily hdf5}
\end{DoxyItemize}

Optional dependencies\+:


\begin{DoxyItemize}
\item \href{http://opencv.org/}{\tt Open\+CV} $>$= 2.\+4 including 3.\+0
\item IO libraries\+: {\ttfamily lmdb}, {\ttfamily leveldb} (note\+: leveldb requires {\ttfamily snappy})
\item cu\+D\+NN for G\+PU acceleration (v7)
\end{DoxyItemize}

Pycaffe and Matcaffe interfaces have their own natural needs.


\begin{DoxyItemize}
\item For Python Caffe\+: {\ttfamily Python 2.\+7} or {\ttfamily Python 3.\+3+}, {\ttfamily numpy ($>$= 1.\+7)}, boost-\/provided {\ttfamily boost.\+python}
\item For M\+A\+T\+L\+AB Caffe\+: M\+A\+T\+L\+AB with the {\ttfamily mex} compiler.
\end{DoxyItemize}

{\bfseries cu\+D\+NN Caffe}\+: for fastest operation Caffe is accelerated by drop-\/in integration of \href{https://developer.nvidia.com/cudnn}{\tt N\+V\+I\+D\+IA cu\+D\+NN}. To speed up your Caffe models, install cu\+D\+NN then uncomment the {\ttfamily U\+S\+E\+\_\+\+C\+U\+D\+NN \+:= 1} flag in {\ttfamily Makefile.\+config} when installing Caffe. Acceleration is automatic. The current version is cu\+D\+NN v7; older versions are supported in older Caffe.

{\bfseries C\+P\+U-\/only Caffe}\+: for cold-\/brewed C\+P\+U-\/only Caffe uncomment the {\ttfamily C\+P\+U\+\_\+\+O\+N\+LY \+:= 1} flag in {\ttfamily Makefile.\+config} to configure and build Caffe without C\+U\+DA. This is helpful for cloud or cluster deployment.

\subsubsection*{C\+U\+DA and B\+L\+AS}

Caffe requires the C\+U\+DA {\ttfamily nvcc} compiler to compile its G\+PU code and C\+U\+DA driver for G\+PU operation. To install C\+U\+DA, go to the \href{https://developer.nvidia.com/cuda-downloads}{\tt N\+V\+I\+D\+IA C\+U\+DA website} and follow installation instructions there. Install the library and the latest standalone driver separately; the driver bundled with the library is usually out-\/of-\/date. {\bfseries Warning!} The 331.$\ast$ C\+U\+DA driver series has a critical performance issue\+: do not use it.

For best performance, Caffe can be accelerated by \href{https://developer.nvidia.com/cudnn}{\tt N\+V\+I\+D\+IA cu\+D\+NN}. Register for free at the cu\+D\+NN site, install it, then continue with these installation instructions. To compile with cu\+D\+NN set the {\ttfamily U\+S\+E\+\_\+\+C\+U\+D\+NN \+:= 1} flag set in your {\ttfamily Makefile.\+config}.

Caffe requires B\+L\+AS as the backend of its matrix and vector computations. There are several implementations of this library. The choice is yours\+:


\begin{DoxyItemize}
\item \href{http://math-atlas.sourceforge.net/}{\tt A\+T\+L\+AS}\+: free, open source, and so the default for Caffe.
\item \href{http://software.intel.com/en-us/intel-mkl}{\tt Intel M\+KL}\+: commercial and optimized for Intel C\+P\+Us, with \href{https://registrationcenter.intel.com/en/forms/?productid=2558}{\tt free} licenses.
\begin{DoxyEnumerate}
\item Install M\+KL.
\item Set up M\+KL environment (Details\+: \href{https://software.intel.com/en-us/node/528499}{\tt Linux}, \href{https://software.intel.com/en-us/node/528659}{\tt OS X}). Example\+: {\itshape source /opt/intel/mkl/bin/mklvars.sh intel64}
\item Set {\ttfamily B\+L\+AS \+:= mkl} in {\ttfamily Makefile.\+config}
\end{DoxyEnumerate}
\item \href{http://www.openblas.net/}{\tt Open\+B\+L\+AS}\+: free and open source; this optimized and parallel B\+L\+AS could require more effort to install, although it might offer a speedup.
\begin{DoxyEnumerate}
\item Install Open\+B\+L\+AS
\item Set {\ttfamily B\+L\+AS \+:= open} in {\ttfamily Makefile.\+config}
\end{DoxyEnumerate}
\end{DoxyItemize}

\subsubsection*{Python and/or M\+A\+T\+L\+AB Caffe (optional)}

\paragraph*{Python}

The main requirements are {\ttfamily numpy} and {\ttfamily boost.\+python} (provided by boost). {\ttfamily pandas} is useful too and needed for some examples.

You can install the dependencies with \begin{DoxyVerb}pip install -r requirements.txt
\end{DoxyVerb}


but we suggest first installing the \href{https://store.continuum.io/cshop/anaconda/}{\tt Anaconda} Python distribution, which provides most of the necessary packages, as well as the {\ttfamily hdf5} library dependency.

To import the {\ttfamily caffe} Python module after completing the installation, add the module directory to your {\ttfamily \$\+P\+Y\+T\+H\+O\+N\+P\+A\+TH} by {\ttfamily export P\+Y\+T\+H\+O\+N\+P\+A\+TH=/path/to/caffe/python\+:\$\+P\+Y\+T\+H\+O\+N\+P\+A\+TH} or the like. You should not import the module in the {\ttfamily caffe/python/caffe} directory!

{\itshape Caffe\textquotesingle{}s Python interface works with Python 2.\+7. Python 3.\+3+ should work out of the box without protobuf support. For protobuf support please install protobuf 3.\+0 alpha (\href{https://developers.google.com/protocol-buffers/}{\tt https\+://developers.\+google.\+com/protocol-\/buffers/}). Earlier Pythons are your own adventure.}

\paragraph*{M\+A\+T\+L\+AB}

Install M\+A\+T\+L\+AB, and make sure that its {\ttfamily mex} is in your {\ttfamily \$\+P\+A\+TH}.

{\itshape Caffe\textquotesingle{}s M\+A\+T\+L\+AB interface works with versions 2015a, 2014a/b, 2013a/b, and 2012b.}

\subsection*{Compilation}

Caffe can be compiled with either Make or C\+Make. Make is officially supported while C\+Make is supported by the community.

\subsubsection*{Compilation with Make}

Configure the build by copying and modifying the example {\ttfamily Makefile.\+config} for your setup. The defaults should work, but uncomment the relevant lines if using Anaconda Python. \begin{DoxyVerb}cp Makefile.config.example Makefile.config
# Adjust Makefile.config (for example, if using Anaconda Python, or if cuDNN is desired)
make all
make test
make runtest
\end{DoxyVerb}



\begin{DoxyItemize}
\item For C\+PU \& G\+PU accelerated Caffe, no changes are needed.
\item For cu\+D\+NN acceleration using N\+V\+I\+D\+IA\textquotesingle{}s proprietary cu\+D\+NN software, uncomment the {\ttfamily U\+S\+E\+\_\+\+C\+U\+D\+NN \+:= 1} switch in {\ttfamily Makefile.\+config}. cu\+D\+NN is sometimes but not always faster than Caffe\textquotesingle{}s G\+PU acceleration.
\item For C\+P\+U-\/only Caffe, uncomment {\ttfamily C\+P\+U\+\_\+\+O\+N\+LY \+:= 1} in {\ttfamily Makefile.\+config}.
\end{DoxyItemize}

To compile the Python and M\+A\+T\+L\+AB wrappers do {\ttfamily make pycaffe} and {\ttfamily make matcaffe} respectively. Be sure to set your M\+A\+T\+L\+AB and Python paths in {\ttfamily Makefile.\+config} first!

{\bfseries Distribution}\+: run {\ttfamily make distribute} to create a {\ttfamily distribute} directory with all the Caffe headers, compiled libraries, binaries, etc. needed for distribution to other machines.

{\bfseries Speed}\+: for a faster build, compile in parallel by doing {\ttfamily make all -\/j8} where 8 is the number of parallel threads for compilation (a good choice for the number of threads is the number of cores in your machine).

Now that you have installed Caffe, check out the \href{gathered/examples/mnist.html}{\tt M\+N\+I\+ST tutorial} and the \href{gathered/examples/imagenet.html}{\tt reference Image\+Net model tutorial}.

\subsubsection*{C\+Make Build}

In lieu of manually editing {\ttfamily Makefile.\+config} to configure the build, Caffe offers an unofficial C\+Make build thanks to , , and other members of the community. It requires C\+Make version $>$= 2.\+8.\+7. The basic steps are as follows\+: \begin{DoxyVerb}mkdir build
cd build
cmake ..
make all
make install
make runtest
\end{DoxyVerb}


See \href{https://github.com/BVLC/caffe/pull/1667}{\tt PR \#1667} for options and details.

\subsection*{Hardware}

{\bfseries Laboratory Tested Hardware}\+: Berkeley Vision runs Caffe with Titan Xs, K80s, G\+TX 980s, K40s, K20s, Titans, and G\+TX 770s including models at Image\+Net/\+I\+L\+S\+V\+RC scale. We have not encountered any trouble in-\/house with devices with C\+U\+DA capability $>$= 3.\+0. All reported hardware issues thus-\/far have been due to G\+PU configuration, overheating, and the like.

{\bfseries C\+U\+DA compute capability}\+: devices with compute capability $<$= 2.\+0 may have to reduce C\+U\+DA thread numbers and batch sizes due to hardware constraints. Brew with caution; we recommend compute capability $>$= 3.\+0.

Once installed, check your times against our \href{performance_hardware.html}{\tt reference performance numbers} to make sure everything is configured properly.

Ask hardware questions on the \href{https://groups.google.com/forum/#!forum/caffe-users}{\tt caffe-\/users group}. 