

 \subsection*{title\+: Arg\+Max Layer }

\section*{Arg\+Max Layer}


\begin{DoxyItemize}
\item Layer type\+: {\ttfamily Arg\+Max}
\item \href{http://caffe.berkeleyvision.org/doxygen/classcaffe_1_1ArgMaxLayer.html}{\tt Doxygen Documentation}
\item Header\+: \href{https://github.com/BVLC/caffe/blob/master/include/caffe/layers/argmax_layer.hpp}{\tt {\ttfamily ./include/caffe/layers/argmax\+\_\+layer.hpp}}
\item C\+PU implementation\+: \href{https://github.com/BVLC/caffe/blob/master/src/caffe/layers/argmax_layer.cpp}{\tt {\ttfamily ./src/caffe/layers/argmax\+\_\+layer.cpp}}
\end{DoxyItemize}

\subsection*{Parameters}


\begin{DoxyItemize}
\item Parameters ({\ttfamily Arg\+Max\+Parameter argmax\+\_\+param})
\item From \href{https://github.com/BVLC/caffe/blob/master/src/caffe/proto/caffe.proto}{\tt {\ttfamily ./src/caffe/proto/caffe.proto}})\+:
\end{DoxyItemize}

\{\% highlight Protobuf \%\} \{\% include proto/\+Arg\+Max\+Parameter.\+txt \%\} \{\% endhighlight \%\} 